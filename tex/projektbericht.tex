\documentclass[12pt, oneside, a4paper]{article}
\usepackage[utf8]{inputenc}
\usepackage[T1]{fontenc}
\usepackage[english]{babel}
\usepackage[babel,german=quotes]{csquotes}

\usepackage{graphicx}
\usepackage{float}

\usepackage[backend=biber, style=ieee]{biblatex}
% \addbibresource{paper.bib}

\usepackage{listings}

\usepackage[left=2.5cm,right=2.5cm,top=2.5cm,bottom=2.5cm]{geometry}
\linespread{1.5}

% höly fucking shit what the fuck
% so just doing backref does not work, but backref=true works
% fyi this makes \ref commands clickable
\usepackage[hidelinks, backref=true]{hyperref} 

\usepackage[nonumberlist,automake=immediate,toc]{glossaries}

\makeglossaries



\newcommand{\projektname}{Modellierung eines Ticketsystems}
\newcommand{\autor}{Emil Schläger}
\newcommand{\matrikelnummer}{2988631}
\newcommand{\betreuer}{Frithard Meyer zu Uptrup}


%% Define custom commands END!

\begin{document}
	\pagenumbering{gobble}
	% Deckblatt
	\begin{center}
		% \href{https://www.intension.de/}{\includegraphics[width=6cm]{images/intension}}\hfill\href{https://www.dhbw-stuttgart.de}{\includegraphics[width=4cm]{images/dhbw}}\\
		\large
		\vspace{3cm}
		\textbf{Semesterprojekt Datenbanken}:\\
		\projektname{}
	\end{center}
	\vfill
	\textbf{Name:}\hfill \autor{}\\
	% \textbf{Matrikelnummer}\hfill \matrikelnummer{}
	\textbf{Kurs:}\hfill INF23B
	
	
	
	\newpage
	\pagenumbering{arabic}
	\section{Anforderungsanalyse}
Es soll eine Ticketplattform entstehen, die es Veranstaltern und Künstlern ermöglicht, Events zu planen und Tickets an Kunden zu verkaufen.\\
Um die Plattform nutzen zu können, muss man sich als User anmelden. Für eine Konto müssen der volle Name, eine Email-Adresse, ein Passwort und eine Kontonummer hinterlegt werden. Angabe einer Telefonnummer für Zwei-Faktor Authentifizierung ist optional. Als User kann man nun Tickets zu Events kaufen. Jedes Ticket besitzt einen Preis, sowie eine Kategorisierung, ob es ein Sitz- oder Stehplatzticket ist. Ist es ein Sitzplatzticket, muss noch die Sitznummer hinterlegt werden. Dabei darf für ein Event selbstverständlich nicht mehrfach derselbe Sitz gebucht werden. Für das jeweilige Event muss gespeichert werden, an welchem Tag es stattfindet, sowie um welche Uhrzeit es anfängt. Es kann auch hinterlegt werden, um wie viel Uhr Einlass ist. Auch findet jedes Event in genau einer Venue statt. Für diese ist jeweils ihr Name, ihre Adresse und ihre Kapazität hinterlegt. Die Kapazität ist hierbei aufgeteilt in die Anzahl an möglichen Steh- und möglichen Sitzplätzen.\\ 
Ein Event ist immer ein Teil einer Tour. Jede Tour ist von einem Veranstalter organisiert, und dementsprechend auch auf der Plattform verwaltet. Z.B. können einzelne Events in größere Venues verlegt werden, wenn das Event ausverkauft ist, aber noch Nachfrage existiert. Zu jeder Tour muss nur der Name gespeichert werden; der Zeitrahmen der Tour errechnet sich durch die Daten des frühesten und spätesten Events der Tour.\\
Auf einer Tour spielt immer mindestens ein Künstler. Oft bringt der Künstler aber Vorbands mit. In diesem Fall ist wichtig zu speichern, in welcher Reihenfolge die Künstler in der Tour spielen. Der Hauptkünstler spielt dabei stets als letztes. Jeder Künstler hat einen Namen und gehört einem Genre an.  Ein Künstler spielt auf jedem Event einer Tour immer dieselbe Setlist. Die Plattform gibt Künstlern die Möglichkeit, diese Setlist auf der Plattform für ihre Fans zu veröffentlichen. Eine Setlist besteht aus mehreren Songs, die wiederum Teil eines Albums sind. Auch diese sind für User der Plattform auf den Profilen der Künstler einzusehen. Ein Album wird immer von einem Label veröffentlicht, auch Künstler sind immer bei einem Plattenlabel unter Vertrag. Für jedes Label muss sein Name, sowie die Adresse seines Hauptsitzes gespeichert werden. Für den Vertrag zwischen Label und Künstler muss auch ein Ablaufdatum gespeichert werden.\\
Die Plattform soll auch die Finanzen einer Tour verwalten können, da die Bezahlung der Tickets bei der Plattform landet.  Organisiert also ein Veranstalter eine Tour für einen Künstler, entsteht dabei ein weiterer Vertrag, der vereinbart, wie viel Prozent der Umsätze der Tour an den Veranstalter gehen. Der Umsatz einer Tour errechnet sich ganz einfach aus der Summe der Preise aller verkauften Tickets. Auch die Plattenlabel verdienen an den Tickets mit. Dieser Festbetrag errechnet sich wie folgt: Bei Veröffentlichung eines Albums $A$ über Label $L$ wird ein Betrag festgelegt, der an das Label gezahlt wird, jedes mal wenn der Künstler einen Song $s \in A$ spielt. Am Ende einer Tour kann also mithilfe der Setlist der Künstler und der Anzahl an Events ausgerechnet werden, wie viel an das Label gezahlt werden muss.\\
Da Künstler, Veranstalter und Label jeweils Zugang auf die Plattform haben müssen, um die für sie relevanten Daten zu verwalten, muss jedem indivuduellen Künstler, Veranstalter oder label ein User Account zugeordnet werden. Damit die Plattform entscheiden kann, welche Zugriffsrechte ein einzelner User hat, wird jedem Account eine entsprechende Rolle zugeordnet.

\section{Entity Relationship Model}
\textit{[INSERT BILD]}


\end{document}
